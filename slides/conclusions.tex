\documentclass[lesson_slides]{subfiles}
%\usepackage{natbib}
%\bibliographystyle{plainnat}
\newcommand{\bibfilename}{mhoRef.bib}
\usepackage{graphicx}
% \graphicspath{ {./images/} }
\usepackage{enumerate}
\usepackage{pifont} % for ding
\usepackage{float} % keeps tables in the exact position they occupy in the code
\usepackage{xcolor} % text colour
\usepackage{gb4e} % leave last

\begin{document}
%%=-=-=-=-=-=-=-=-=-=-=-=-=-=-=-=-=-=-=-=-=-=-=-=-=-=-=-=-=-=-=-=-=-=-=-=-=-=-=-=
%   FRAME START   -=-=-=-=-=-=-=-=-=-=-=-=-=-=-=-=-=-=-=-=-=-=-=-=-=-=-=-=-=-=-=
\begin{frame}[c]{Conclusions}

        \transboxin<1>
        \transglitter<2>
        \transwipe<3>
        \begin{itemize}
        \item[\ding{227}] French wh-interrogatives have evolved in the direction of no IM, no SO and no IM\textsubscript{lex}, as per our working hypotheses; \pause
        \item[\ding{227}] the parametrization of the projection responsible for interrogative wh-movement in French is evolving towards IM=0 and is at present at the optionality stage seen for Heian Chinese (IM=0/1).  \pause
        \item[\ding{227}] \vspace*{-2mm} Rizzi's (2017) elegant understanding of parameters allows for fine descriptions of how functional projections work, and evolve over time.
        \end{itemize}
  
\end{frame}
%   FRAME END   --==-=-=-=-=-=-=-=-=-=-=-=-=-=-=-=-=-=-=-=-=-=-=-=-=-=-=-=-=-=-=
%   FRAME START   -=-=-=-=-=-=-=-=-=-=-=-=-=-=-=-=-=-=-=-=-=-=-=-=-=-=-=-=-=-=-=
\begin{frame}[c]{}

\begin{center}
    \huge{Thank you for your attention!}
\end{center}
  
\end{frame}
%   FRAME END   --==-=-=-=-=-=-=-=-=-=-=-=-=-=-=-=-=-=-=-=-=-=-=-=-=-=-=-=-=-=-=
%=-=-=-=-=-=-=-=-=-=-=-=-=-=-=-=-=-=-=-=-=-=-=-=-=-=-=-=-=-=-=-=-=-=-=-=-=-=-=-=
\end{document}