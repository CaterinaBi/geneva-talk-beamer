\documentclass[lesson_slides]{subfiles}
%\usepackage{natbib}
%\bibliographystyle{plainnat}
\newcommand{\bibfilename}{mhoRef.bib}
\usepackage{graphicx}
% \graphicspath{ {./images/} }
\usepackage{enumerate}
\usepackage{pifont} % for ding
\usepackage{float} % keeps tables in the exact position they occupy in the code
\usepackage{xcolor} % text colour
\usepackage{gb4e} % leave last

\begin{document}
%%=-=-=-=-=-=-=-=-=-=-=-=-=-=-=-=-=-=-=-=-=-=-=-=-=-=-=-=-=-=-=-=-=-=-=-=-=-=-=-=
%   FRAME START   -=-=-=-=-=-=-=-=-=-=-=-=-=-=-=-=-=-=-=-=-=-=-=-=-=-=-=-=-=-=-=
\begin{frame}[c]{What?}

    \transboxin<1>
    \transglitter<2>
    \transwipe<3>
    \textbf{\textsc{this paper}} \pause
    \begin{itemize}
        \item[\ding{227}] wh-interrogatives in French; \pause
        \item[\ding{227}] micro-diachronic perspective \pause (ESLO 1-2 corpora, 1970s-2010s);\\ \pause
        \item[\ding{227}] wh-in situ \pause ('T'as vu qui?').
    \end{itemize}
    
\end{frame}
%   FRAME END   --==-=-=-=-=-=-=-=-=-=-=-=-=-=-=-=-=-=-=-=-=-=-=-=-=-=-=-=-=-=-=
%=-=-=-=-=-=-=-=-=-=-=-=-=-=-=-=-=-=-=-=-=-=-=-=-=-=-=-=-=-=-=-=-=-=-=-=-=-=-=-=
%%=-=-=-=-=-=-=-=-=-=-=-=-=-=-=-=-=-=-=-=-=-=-=-=-=-=-=-=-=-=-=-=-=-=-=-=-=-=-=-=
%   FRAME START   -=-=-=-=-=-=-=-=-=-=-=-=-=-=-=-=-=-=-=-=-=-=-=-=-=-=-=-=-=-=-=
\begin{frame}[c]{Why?}

    \transboxin<1>
    \transglitter<2>
    \transwipe<3>
    \textbf{\textsc{why wh-in situ (again)?}} \pause
    \begin{itemize}
        \item[\ding{227}] very controversial topic; \pause
        \item[\ding{227}] felicity goes beyond syntax alone (interpretation, prosody, etc.); \pause
        \item[\ding{227}] is wh-in situ strogly presuppositional, presuppositional or not necessarily so?\\ \pause 
        $\longrightarrow$ we might have asked the wrong question all along!
    \end{itemize}
  
\end{frame}
%   FRAME END   --==-=-=-=-=-=-=-=-=-=-=-=-=-=-=-=-=-=-=-=-=-=-=-=-=-=-=-=-=-=-=
%   FRAME START   -=-=-=-=-=-=-=-=-=-=-=-=-=-=-=-=-=-=-=-=-=-=-=-=-=-=-=-=-=-=-=
\begin{frame}[c]{What's the right question, then?}

    \transboxin<1>
    \transglitter<2>
    \transwipe<3>
    \textbf{\textsc{levels of activation}} \pause
    \begin{itemize}
    \item[\ding{227}] Larrivé (2019) developed an interesting framework for the understanding of the interpretation of wh-in situ which relies on the notion of 'activation'; \pause (no more presupposition) \pause 
    \item[\ding{227}] Larrivé thus made a \textbf{bipartite} classification of French wh-in situ based on this concept;\\ \pause
    $\longrightarrow$ explicitly activated vs. non-explicitely activated. \pause
    \item[\ding{227}] Garassino (2021) refines this bipatition and delivers a \textbf{tripartition};\\ \pause
    $\longrightarrow$ explicitly activated vs. non-explicitely activated vs. inferred. \pause
    \end{itemize}

    This tripartition is powerful in predicting when these structures are licensed, and how they evolve.
  
\end{frame}
%   FRAME END   --==-=-=-=-=-=-=-=-=-=-=-=-=-=-=-=-=-=-=-=-=-=-=-=-=-=-=-=-=-=-=

%   FRAME START   -=-=-=-=-=-=-=-=-=-=-=-=-=-=-=-=-=-=-=-=-=-=-=-=-=-=-=-=-=-=-=
\begin{frame}[c]{What do we claim?}

    \transboxin<1>
    \transglitter<2>
    \transwipe<3>
    \textbf{\textsc{what is our claim?}}
    \begin{itemize}
    \item[\ding{227}] we re-did Larrivée study (heavily sociolinguistic) based on syntax-informed criteria and Garassino's tripartition.
        \item[\ding{227}] our study confirms Larrivée's (2019) claim; \pause (although with quite different data); \pause
        \item[\ding{227}] Larrivé: wh-in situ in French is quite scarce until a few decades into the 20\textsuperscript{th} century, and merely \textbf{explicitly activated} \pause ($=$ heavily context dependant) \pause
        \item[\ding{227}] Larrivée \& our study: wh-in situ evolves to be felicitous also in \textbf{non-activated} contexts \pause ($=$ context independent). \pause
        \item[\ding{227}] Our study: also context-inferrable occurrences increase over time\pause, at the expenses of context-dependency.
    \end{itemize}
\end{frame}

\end{frame}
%   FRAME END   --==-=-=-=-=-=-=-=-=-=-=-=-=-=-=-=-=-=-=-=-=-=-=-=-=-=-=-=-=-=-=
%=-=-=-=-=-=-=-=-=-=-=-=-=-=-=-=-=-=-=-=-=-=-=-=-=-=-=-=-=-=-=-=-=-=-=-=-=-=-=-=
\end{document}