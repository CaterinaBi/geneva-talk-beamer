\documentclass[lesson_slides]{subfiles}
%\usepackage{natbib}
\usepackage{graphicx}
% \graphicspath{ {./images/} }
\usepackage{enumerate}
\usepackage{pifont} % for ding
\usepackage{float} % keeps tables in the exact position they occupy in the code
\usepackage{xcolor} % text colour
\usepackage{gb4e} % leave last

\begin{document}
%%=-=-=-=-=-=-=-=-=-=-=-=-=-=-=-=-=-=-=-=-=-=-=-=-=-=-=-=-=-=-=-=-=-=-=-=-=-=-=-=
%   FRAME END   --==-=-=-=-=-=-=-=-=-=-=-=-=-=-=-=-=-=-=-=-=-=-=-=-=-=-=-=-=-=-=
\begin{frame}[c]{Wh-questions}

    \transboxin<1>
    \transglitter<2>
    \transwipe<3>
    \noindent \textbf{\textsc{genuine wh-questions}} \pause (answer-seeking, as opposed to echo etc.)\\ \pause
    $\longrightarrow$ (non-genuine questions are interpretationally and often syntactically different, e.g. they can trigger wh-in situ also in wh-fronting languages like English etc.)
    \begin{itemize}
        \item[\ding{227}] bear on a wh-element (qui, quoi, comment, etc.); \pause
        \item[\ding{227}] require an answer that bears on that same wh-element. \pause 
        \item[\ding{227}] have different shapes \pause (cross-linguistically and language-internally!).
    \end{itemize}
\end{frame}
%   FRAME END   --==-=-=-=-=-=-=-=-=-=-=-=-=-=-=-=-=-=-=-=-=-=-=-=-=-=-=-=-=-=-=
%   FRAME START   -=-=-=-=-=-=-=-=-=-=-=-=-=-=-=-=-=-=-=-=-=-=-=-=-=-=-=-=-=-=-=
\begin{frame}[c]{Question-formation strategies in French}

    \transboxin<1>
    \transglitter<2>
    \transwipe<3>
    \noindent \textbf{\textsc{french wh-questions}} \pause
    \begin{itemize}
        \item[\ding{227}] in situ: “Tu as vu qui?”; \pause
        \item[\ding{227}] ex situ: “Qui tu as vu?”; \pause 
        \item[\ding{227}] ex situ: “Qui est-ce que tu as vu?”; \pause 
        \item[\ding{227}] ex situ: “Qui as-tu vu?”; \pause 
        \item[\ding{227}] cleft: “C’est qui que tu as vu?”.
    \end{itemize}
   
\end{frame}
%   FRAME END   --==-=-=-=-=-=-=-=-=-=-=-=-=-=-=-=-=-=-=-=-=-=-=-=-=-=-=-=-=-=-=
%   FRAME START   -=-=-=-=-=-=-=-=-=-=-=-=-=-=-=-=-=-=-=-=-=-=-=-=-=-=-=-=-=-=-=
\begin{frame}[c]{Question-formation strategies in French}

    \noindent \textbf{\textsc{french wh-questions}}
    \begin{itemize}
        \item[\ding{227}] in situ: “Tu as vu \hl{qui}?”;
        \item[\ding{227}] ex situ: “Qui tu as vu?”;
        \item[\ding{227}] ex situ: “Qui est-ce que tu as vu?”; 
        \item[\ding{227}] ex situ: “Qui as-tu vu?”;
        \item[\ding{227}] cleft: “C’est qui que tu as vu?”.
    \end{itemize}
   
\end{frame}
%   FRAME END   --==-=-=-=-=-=-=-=-=-=-=-=-=-=-=-=-=-=-=-=-=-=-=-=-=-=-=-=-=-=-=
%   FRAME START   -=-=-=-=-=-=-=-=-=-=-=-=-=-=-=-=-=-=-=-=-=-=-=-=-=-=-=-=-=-=-=
\begin{frame}[c]{Question-formation strategies in French}

    \noindent \textbf{\textsc{french wh-questions}}
    \begin{itemize}
        \item[\ding{227}] in situ: “Tu as vu \hl{qui}?”;
        \item[\ding{227}] ex situ: “\hl{Qui} tu as vu?”;
        \item[\ding{227}] ex situ: “\hl{Qui} est-ce que tu as vu?”; 
        \item[\ding{227}] ex situ: “\hl{Qui} as-tu vu?”;
        \item[\ding{227}] cleft: “C’est qui que tu as vu?”.
    \end{itemize}
   
\end{frame}
%   FRAME END   --==-=-=-=-=-=-=-=-=-=-=-=-=-=-=-=-=-=-=-=-=-=-=-=-=-=-=-=-=-=-=
%   FRAME START   -=-=-=-=-=-=-=-=-=-=-=-=-=-=-=-=-=-=-=-=-=-=-=-=-=-=-=-=-=-=-=
\begin{frame}[c]{Question-formation strategies in French}

    \noindent \textbf{\textsc{french wh-questions}}
    \begin{itemize}
        \item[\ding{227}] in situ: “Tu as vu \hl{qui}?”;
        \item[\ding{227}] ex situ: “\hl{Qui}\textsubscript{i} tu as vu \_\_\_\textsubscript{i}?”;
        \item[\ding{227}] ex situ: “\hl{Qui}\textsubscript{i} est-ce que tu as vu \_\_\_\textsubscript{i}?”; 
        \item[\ding{227}] ex situ: “\hl{Qui}\textsubscript{i} as-tu vu \_\_\_\textsubscript{i}?”;
        \item[\ding{227}] cleft: “C’est qui que tu as vu?”.
    \end{itemize}
   
\end{frame}
%   FRAME END   --==-=-=-=-=-=-=-=-=-=-=-=-=-=-=-=-=-=-=-=-=-=-=-=-=-=-=-=-=-=-=
%   FRAME START   -=-=-=-=-=-=-=-=-=-=-=-=-=-=-=-=-=-=-=-=-=-=-=-=-=-=-=-=-=-=-=
\begin{frame}[c]{Question-formation strategies in French}

    \noindent \textbf{\textsc{french wh-questions}}
    \begin{itemize}
        \item[\ding{227}] in situ: “Tu as vu \hl{qui}?”; $\longrightarrow$ SV
        \item[\ding{227}] ex situ: “\hl{Qui}\textsubscript{i} tu as vu \_\_\_\textsubscript{i}?”;
        \item[\ding{227}] ex situ: “\hl{Qui}\textsubscript{i} est-ce que tu as vu \_\_\_\textsubscript{i}?”; 
        \item[\ding{227}] ex situ: “\hl{Qui}\textsubscript{i} as-tu vu \_\_\_\textsubscript{i}?”;
        \item[\ding{227}] cleft: “C’est qui que tu as vu?”.
    \end{itemize}
   
\end{frame}
%   FRAME END   --==-=-=-=-=-=-=-=-=-=-=-=-=-=-=-=-=-=-=-=-=-=-=-=-=-=-=-=-=-=-=
%   FRAME START   -=-=-=-=-=-=-=-=-=-=-=-=-=-=-=-=-=-=-=-=-=-=-=-=-=-=-=-=-=-=-=
\begin{frame}[c]{Question-formation strategies in French}

    \transboxin<1>
    \transglitter<2>
    \transwipe<3>
   
\end{frame}
%   FRAME END   --==-=-=-=-=-=-=-=-=-=-=-=-=-=-=-=-=-=-=-=-=-=-=-=-=-=-=-=-=-=-=
%=-=-=-=-=-=-=-=-=-=-=-=-=-=-=-=-=-=-=-=-=-=-=-=-=-=-=-=-=-=-=-=-=-=-=-=-=-=-=-=
\begin{frame}[c]{Question-formation strategies in French}

    \transboxin<1>
    \transglitter<2>
    \transwipe<3>
   
\end{frame}
%   FRAME END   --==-=-=-=-=-=-=-=-=-=-=-=-=-=-=-=-=-=-=-=-=-=-=-=-=-=-=-=-=-=-=
\begin{frame}[c]{Question-formation strategies in French}

    \transboxin<1>
    \transglitter<2>
    \transwipe<3>
   
\end{frame}
%   FRAME END   --==-=-=-=-=-=-=-=-=-=-=-=-=-=-=-=-=-=-=-=-=-=-=-=-=-=-=-=-=-=-=
%=-=-=-=-=-=-=-=-=-=-=-=-=-=-=-=-=-=-=-=-=-=-=-=-=-=-=-=-=-=-=-=-=-=-=-=-=-=-=-=
\end{document}